\section{Background}

\subsection{General architecture of document management systems}

A document management system is a typical client-server model: a document
management server stores the documents in a document repository, which can be
accessed via various interfaces.  The other part of the system is a client,
which has built-in support for opening, saving and editing documents.

\begin{figure}[H]
\centering
\includegraphics[width=450px,keepaspectratio]{general-arch-of-doc-mgmt-systems.pdf}
\caption{Architecture of a document management system}
\label{fig:general-arch-of-doc-mgmt-systems}
\end{figure}

These two entities of the system (see
figure~\ref{fig:general-arch-of-doc-mgmt-systems}) have the following
operations:

\begin{itemize}
\item The server primarily listens to client requests. Additionally it allows
performing operations directly on the server such as registering new users or
sharing documents of a user to others.
\item The client, connecting to one or more document server.
\end{itemize}

Every user can have access to document \textbf{workspaces}.\footnote{Users and workspaces are in an N:N relation: a user can have access to multiple workspaces and multiple users can use the same workspace.} Workspaces can have documents,
links and tasks. A workspace can be shared with different permissions
(read-only, read-write), and that is typically done by sending an invitation
email which the other user can accept.

A user may access the document server using a web browser, or via rich client
applications. The advantage of the web browser interface is that it can be
accessed from almost everywhere, however, document editing can't be performed.
If such operation should be performed, then the user has to manually download
the document, edit then upload it. This method occasionally does not cause a
problem, but of course it's uncomfortable for daily work.

The other interface is a rich client, which is installed on the machine of the
user. Vendors prefer to produce a corresponding client for their server,
Microsoft SharePoint and Microsoft Office is a typical setup.

In case of servers or clients speaking different communication protocols,
selection of the used protocol is selected differently on client and server
side. Servers can listen on different addresses, and in this case the address
identifies not only the server, but the used protocol as well.

For example Alfresco, has its native protocol, but also (more or less) provides
support for the SharePoint protocol. As a result, it can be configured to
listen on one URL as an Alfresco server, and on an other URL as a SharePoint
server. Additionally, it even supports CIFS, FTP and NFS for some
operations.\cite{alfresco-fsc}

Clients can have different extensions or plug-ins to handle different
protocols. For example Microsoft Office can accept SharePoint URL-s in the
standard file opener dialog, while the Alfresco extension for OpenOffice.org
has a dedicated menu in the application to connect to an Alfresco server.

It is also common that the client extensions have minimal business logic. For
example the proprietary SharePoint extension to OpenOffice.org, created by
Oracle can't talk to every SharePoint server, but Microsoft Office does -- as
long as a server-side component provided by Oracle is not installed on the
server. While this approach may be compelling during development, actually it
is uncomfortable for system administrators.

\subsection{Related standards}

\begin{figure}[H]
\centering
\includegraphics[width=250px,keepaspectratio]{cmis.pdf}
\caption{Architecture of the CMIS standard}
\label{fig:cmis}
\end{figure}

The specification called Content Management Interoperability
Services\cite{cmis} is created to address compatibility issues between various
document management server implementations. OASIS approved it on May 1, 2010.
Also, at the time of writing implementation is far from complete among major
document management servers. Alfresco 3.3+ implements the client side only,
SharePoint 2007 does not support it yet, so at the moment it's a vision that
all major document servers will implement this specification.

On the other hand, there are a lot of implementations in other products, such
as IBM Lotus Connections 3.0\cite{lotus} (server side only) or TYPO3 (client
and server side).

CMIS has two main goals:

\begin{itemize}
\item Providing a list of web service (SOAP) interfaces, it's language-independent.
\item It allows separating the service and content, making it possible to
implement services for legacy document repositories without modifying them.
\end{itemize}

To achieve this, it defines a domain model, describing the following data model
elements (see figure~\ref{fig:cmis}):

\begin{itemize}
\item repository: A container for objects.
\item object: Common ancestor for document, folder, relationship and policy objects.
\item object-type: A set of properties, all instances of a type have those.
\item document object: The document entities managed by the repository.
\item folder object: Container for file objects.
\item relationship object: Representing links between objects.
\item policy object: Policies enforced by a repository.
\item access control: Consists of pre-defined permissions and allows defining new ones.
\item versioning: Versions document objects (other object types are not allowed).
\item query: Type-based query to list objects matching a specified criteria.
\item change log: Changes to objects since a given timestamp.
\end{itemize}

Then -- using this data model -- it specifies services to operate on them:

\begin{itemize}
\item common service elements: Includes paging, exception handling.
\item repository services
\item navigation services
\item object services: Includes factory functions for different object types.
\item multi-filing services: Makes it possible to add/remove objects to folders.
\item discovering services: To search query-able objects within a repository.
\item versioning services: Includes check in, check out, cancel checkout, get
	versions.
\item relationship services: Provides the relationships of an object.
\item policy services: Apply and remove policies, list applied ones.
\item ACL services: Get and apply an ACL.
\end{itemize}

Finally it explains concrete syntaxes: the Restful AtomPub Binding and the Web
Services (SOAP) binding.

\subsection{Concrete implementations}

In this subsection, we give a short comparision of Microsoft SharePoint 2007 and
Alfresco Community Edition 3.4, which claims to be "The Open Source Alternative
for Enterprise Content Management".

A general notable difference is that Alfresco does not allow deletion where
possible. Of course it is possible to delete documents, but it isn't possible
to delete document versions, overwrite a previous document version or delete a
whole workspace as a single operation.

On the other hand, SharePoint uses its leading position to avoid publishing
proper documentation on how communication protocol it uses works. This protocol
existed before the CMIS standard was published, and it is not a standard. For
example the \emph{put document} method has a \emph{document} parameter -- the
documentation\cite{spdoc} does not mention the \emph{timestamp of last
modification}, part of the meta information structure, which is in fact
mandatory (and there is no problem with that being mandatory, since it uses an
optimistic concurrency approach). There is no similar problem with Alfresco,
where in worst case it's possible to check the source code.

When evaluating properties, we selected the most common use-cases, required by
enterprise companies. We intentionally did not evaluate management of document
workspace permissions, links inside document workspaces and tasks inside
document workspaces, as those features are outside of the scope of the current
thesis.

Individual properties we collected are detailed in
table~\ref{tab:background-comparison}.

\begin{table}[H]
\begin{threeparttable}
  \begin{center}
    \begin{tabular}{| l | l | l |}
    \hline
    \textbf{Feature} & \textbf{SharePoint} & \textbf{Alfresco} \\ \hline
    License & proprietary & open-source \\ \hline
    Maturity & 14 years\cite{sphist} & 6 years \\ \hline
    Creating a document workspace & supported & supported \\ \hline
    Deleting a document workspace & supported & not supported\tnote{1}\\ \hline
    Checking out a document & supported & supported \\ \hline
    Checking in a document & supported & supported \\ \hline
    Cancelling the checkout of a document & supported & supported \\ \hline
    Getting documents & supported & supported \\ \hline
    Putting documents & supported & supported \\ \hline
    Listing versions of a document & supported & supported \\ \hline
    Viewing previous versions a document & supported & supported \\ \hline
    Deleting previous versions a document & supported & not supported \\ \hline
    Overwriting previous versions a document & supported & not supported \\ \hline
    Restoring a previous versions a document & supported & supported \\ \hline
    Deleting a document & supported & supported \\ \hline
    \end{tabular}
    \begin{tablenotes}
    \item [1] This feature is supported when using the native protocol of Alfresco.
    \end{tablenotes}
  \end{center}
  \caption{Comparison of SharePoint and Alfresco}
  \label{tab:background-comparison}
\end{threeparttable}
\end{table}

\subsection{Differences to version control systems}

Version control systems in general have richer semantics, more features to
track software source code. In order to make document management systems easy
to use for non-technical people, some of the features of version control
systems are simply missing from document management systems. In this
subsection, we describe the major removed operations.

We note that this does not mean version control systems are superior. They are
optimized to handle source code, which is plain text. Version control systems
are not efficient in handling binary files -- for example zipped XML files, used by
the ODF/OOXML formats -- while document management systems are designed to deal
with such files.

The development of source code is typically not linear. It's regular that
branches are created and merged during development, while a document server
forces you to follow a single development line.

Source code repositories are also checked out as a single operation,
containing all files of the repository. Similarly, when a client checks in
multiple files, that makes a single commit, and later it is possible to see all
the files modified by that commit. Document servers make the assumption that
the client wants to check out a single file, and a commit affects only a single
file.

Documents are improving, but we rarely speak about document bugs, as we speak
about software bugs. Because of that, version control systems usually provide a
way to check out the state of the entire repository at a given earlier
time, to discover which commit introduced a specific bug. Document servers
allow listing of versions of a document, then the user manually has to select
the version which is closest to a given date.

Finally, version control systems provide a way to annotate source code files:
to determine the author of every single line of a file, where the author of a
line is the person who authored the commit introducing the line in question.
Document management servers do not pay attention to this, since the document
formats they usually track (think of ODF or OOXML, again) provide a \emph{track
changes} feature already, so it makes little sense to duplicate that
functionality on the server side.\footnote{Not to mention that these document
formats store the \emph{model} of the document (model, as in
Model-View-Controller), and when we speak about lines, we speak about the
\emph{view} of the model, so the same annotate operation would have a different
meaning here.}

To sum up, we can see that document management systems feature less operations
in general to make usage by wider user base possible -- but we should not
forget that version control systems solve a related, but still different
problem, so either of them is an ultimate tool, making the other one useless.

\subsection{UNO compontents}

My solution will be a LibreOffice extension that registers an UNO\cite{uno}
component. UNO is the interface based component model of LibreOffice. The main
feature it provides is that interfaces are defined in a language-idependent
IDL-like syntax, and they can be implemented in multiple languages:

\begin{itemize}
\item C
\item C++
\item Python
\item Java
\item BASIC
\end{itemize}

The caller is not aware which language the implementation uses, and it does not
need to, either. In our extension, we call BASIC macros from the menu elements,
which invoke the underlying Java code, where the business logic is implemented.

All the Java code is contained by a single jar file, the workflow of creating
it is shown on figure~\ref{fig:uno-java}.

\begin{figure}[H]
\centering
\includegraphics[width=250px,keepaspectratio]{uno-java.pdf}
\caption{Workflow of implementing an UNO service in Java}
\label{fig:uno-java}
\end{figure}

\begin{itemize}
\item \emph{idlc} compiles the .idl files to UNO Resource Descriptors,
\item \emph{regmerge} merges all the .urd files of the component to a single Resource Database,
\item \emph{javamaker} creates interfaces (.class files) from the Resource Database,
\item finally the standard Java development tools (\emph{javac}, \emph{jar})
creates the final jar, including the Java implementation.
\end{itemize}

Once the jar package is ready, a zip archive is created, containing:

\begin{itemize}
\item \emph{Addons.xcu}, containing the menu items
\item the BASIC library (callbacks for the menu items)
\item Java libraries used by the Java implementation
\item the Java implementation
\item metamodel of the stored settings
\item other files: metadata, extension description, license, etc.
\end{itemize}

Now that we saw the concept of UNO components and the structure of an
extension, the rest of this section describes the design of the UNO
specifications and Java implementations provided by the extension.

