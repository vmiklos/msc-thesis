\section{Introduction}

Nobody wants to loose work. Version control is a great idea to address this
problem: users of the system are required to mark consistent states of the
system, and then those states are committed to some kind of storage. Whenever
the user realises there is a need to retrieve an older version of the artifact,
the system can provide it.

There are many different types of version control software. One widely known
class of those is the one used by software developers: they are optimized to
track the source code of a project. One class of those works in a centralized
manner (Subversion\cite{subversion} is probably the most popular centralized
solution), and there are many other version control projects which operate even
in a distributed environment (Git\cite{git} is a good such example).

An other type of similar software is usually referred as document management
system. It focuses on project management, and because of that, it targets not
only developers, but less technical users. That means usually it manages not
source code files, but documents created by office suites such as Microsoft
Office or LibreOffice.

Managing such documents requires different operations, for example it makes
little sense to make a document management server distributed -- on the other
hand one would rarely send invitation links to retrieve source code files.
Document management servers are rarely used by small companies, it's a typical
enterprise product: it's optimised for large user base, and hardware
requirements are usually not a problem, it's possible to dedicate a standalone
machine to provide document management facilities.

It's important to realise how important a GUI is for these users. A natural
need is to access the document management operations directly from the office
software, rather than saving documents to the local disk, opening a web browser
and uploading or downloading files manually.

Every enterprise environment consists of many products, usually produced by
different vendors. Vendors come and go, and companies try to avoid vendor
lock-in when choosing between alternative software products. As a consequence,
the license of the products and of the communication protocols is a major
factor during the evaluation of these products.

The open-source licenses and development model plays an important role here. As
long as a company receives source code with the binary of the product they use,
they can invest time or money to the development of that software, which is not
possible with closed source products. On the other hand, if a company plans to
replace only a single component of a system to an open-source alternative, they
need to make sure the new component is able to communicate with the old
proprietary components.

It is not the aim of this introduction to specify a feature list for an office
suite to become enterprise-ready, but I want to point out the need of
enterprise support in open-source office suites, which are clearly needs
improving in the near future.

FIXME A short general introduction about document workflows.

The rest of this introduction is structured into two subsections. The first
gives a brief introduction to already existing document management systems,
while the second will introduce document workflow engines.

\subsection{Document management}

FIXME A bit longer introduction to document management.

\subsection{Document workflows}

FIXME A bit longer introduction to document workflows.
